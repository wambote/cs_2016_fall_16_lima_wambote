\documentclass{article}
\usepackage[utf8]{inputenc}

\title{Technology}
\author{wambote.lima }
\date{December 2016}

\usepackage{natbib}
\usepackage{graphicx}

\begin{document}

\maketitle

\section{Introduction}
Technology science of craft, from Greek , techne, art, skill, cunning of hand; and , -logia 2 is the collection of techniques, skills, methods and processes used in the production of goods or services or in the accomplishment of objectives, such as scientific investigation. 

\begin{figure}[!htb]
    \centering
    \includegraphics[scale=0.40]{technology.jpg}
    \caption{The Technology}
    \label{fig:technology}
\end{figure}

\section{Development}
Technology can be the knowledge of techniques, processes, and the like, or it can be embedded in machines, computers, devices, and factories, which can be operated by individuals without detailed knowledge of the workings of such things.The use of the term "technology" has changed significantly over the last 200 years. Before the 20th century, the term was uncommon in English, and usually referred to the description or study of the useful arts.[3] The term was often connected to technical education, as in the Massachusetts Institute of Technology (chartered in 1861).

\section{Conclusion}
Theories of technology often attempt to predict the future of technology based on the high technology and science of the time. 

\bibliographystyle{plain}
\bibliography{references}
\end{document}
